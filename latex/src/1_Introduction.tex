\chapter{Giới thiệu nghiên cứu}
\section{Giới thiệu đề tài}
\tab Trong thời đại công nghệ 4.0 hiện nay, có rất nhiều phát minh, sản phẩm công nghệ mới đã ra đời nhờ sự nghiên cứu sáng tạo của con người, bao gồm Internet of Things (IOTs), công nghệ Blockchain, thực tế ảo (Virtual Reality - VR),.... Trong số đó, không thể không kể đến lĩnh vực robot và tự động hóa. Robot tự hành đang trở thành xu hướng công nghệ trong tương lai vì sự tiện dụng, khả năng hoạt động chính xác trong nhiều điều kiện khắc nghiệt mà con người không thể làm việc. Một trong những yêu cầu cơ bản đối với lĩnh vực này là robot có khả năng hoạt động theo chức năng được lập trình sẵn mà không cần đến sự can thiệp của con người.\\
\tab Có rất nhiều vấn đề đặt ra cho robot tự hành, tuy nhiên, robot tự hành bám làn đường là một vấn đề cơ bản trong số đó. Robot có khả năng bám làn đường giúp giảm nguy cơ tai nạn giao thông do sai sót của con người, đồng thời tăng cường khả năng dự báo và phản ứng của hệ thống. Hệ thống bám làn đường này có thể được tích hợp vào các phương tiện giao thông thông thường để cải thiện chất lượng di chuyển và giảm thiểu tác động tiêu cực đến môi trường.\\
\tab Ngoài ra, với tầm nhìn xa hơn, đề tài này cũng mang lại nhiều tiềm năng trong việc phát triển các hệ thống vận chuyển thông minh. Việc có những robot có khả năng tự lái trên đường sá không chỉ giúp tối ưu hóa quy trình di chuyển mà còn mở ra cánh cửa cho những ứng dụng mới, từ vận chuyển hàng hóa đến dịch vụ di động.\\
\tab Chính vì thế, nhóm chọn đề tài "Hiện thực chức năng bám làn đường cho robot tự hành", đề tài không chỉ giúp giải quyết các vấn đề hiện tại về giao thông và an toàn, mà còn mang lại những triển vọng tương lai hứa hẹn trong lĩnh vực vận chuyển và tự động hóa.
\section{Mục tiêu}
\tab Mục tiêu của nhóm là hiện thực các chức năng sao cho robot có thể tự hoạt động (tự di chuyển) trong nhiều môi trường nhất có thể (từ môi trường lý tưởng trong trình mô phỏng cho đến môi trường thực tế có nhiễu). Các chức năng được xem xét để hiện thực như sau:
\begin{itemize}
    \item Có khả năng nhận diện được làn đường.
    \item Có khả năng di chuyển vào tâm làn đường.
    \item Đưa ra cảnh báo khi phát hiện nguy cơ lệch khỏi làn đường.
\end{itemize}
\section{Phạm vi đề tài}
Trước những khó khăn chung của ngành công nghiệp xe tự hành, và sự đa dạng các trường hợp từ môi trường thực tế phức tạp, nhóm phải giới hạn lại phạm vi đề tài nhằm đảm bảo đúng tiến độ và khả năng thực thi. Giới hạn của đề tài bao gồm những điểm sau:
\begin{itemize}
    \item \textbf{Nhận diện làn đường:} Robot có khả năng di chuyển trong làn đường thẳng và làn đường cong, có các tính chất sau:
        \begin{itemize}
            \item Làn đường nét liền.
            \item Làn đường nét đứt.
            \item Làn đường có thể có vật thể chắn tối đa 50\% chiều dài của làn đường.
            \item Không bao gồm trường hợp chuyển làn đường.
        \end{itemize}
    \item \textbf{Tốc độ:} 
        \begin{itemize}
            \item Tốc độ phản hồi tối thiểu của việc nhận diện làn đường là 100ms.
            \item Tốc độ di chuyển tối đa của robot là 20cm/s.
        \end{itemize}
\end{itemize}
\section{Ý nghĩa thực tiễn}
\tab Đề tài "Hiện thực chức năng bám làn đường cho robot tự hành" đang mang lại những ý nghĩa thực tiễn vô cùng quan trọng trong thời đại hiện nay, đặc biệt trong bối cảnh xã hội đang chuyển đổi hướng vào Công nghệ 4.0. Nhóm hi vọng nghiên cứu của nhóm sẽ phần nào đóng góp cho các mô hình áp dụng xe tự hành. Các mô hình có thể áp dụng chức năng bám làn đường cho robot tự hành như sau:
\begin{itemize}
    \item \textbf{Phương tiện di chuyển:} Áp dụng hệ thống bám làn đường cho các xe tự hành giúp giảm áp lực và mệt mỏi cho người lái xe, tạo điều kiện cho họ tập trung vào các hoạt động khác trong quá trình di chuyển. Ngoài ra, xe tự hành được trang bị hệ thống bám làn đường có thể trở thành một giải pháp hữu ích cho người khuyết tật và người cao tuổi, giúp họ dễ dàng di chuyển mà không phụ thuộc vào sự hỗ trợ của người khác.
    \item \textbf{Robot vận chuyển:} Ứng dụng robot tự hành bám làn đường giúp giảm chi phí và tăng hiệu suất trong quá trình vận chuyển hàng hóa. Ví dụ như, tại các nhà máy, công xưởng tự động cao, nơi mà các robot sẽ di chuyển theo đường đi vẽ sẵn để có thể phục vụ việc vận chuyển vật tư, thiết bị.
\end{itemize}
